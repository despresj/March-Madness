% Options for packages loaded elsewhere
\PassOptionsToPackage{unicode}{hyperref}
\PassOptionsToPackage{hyphens}{url}
\PassOptionsToPackage{dvipsnames,svgnames*,x11names*}{xcolor}
%
\documentclass[
  english,
  man,floatsintext]{apa6}
\usepackage{amsmath,amssymb}
\usepackage{lmodern}
\usepackage{ifxetex,ifluatex}
\ifnum 0\ifxetex 1\fi\ifluatex 1\fi=0 % if pdftex
  \usepackage[T1]{fontenc}
  \usepackage[utf8]{inputenc}
  \usepackage{textcomp} % provide euro and other symbols
\else % if luatex or xetex
  \usepackage{unicode-math}
  \defaultfontfeatures{Scale=MatchLowercase}
  \defaultfontfeatures[\rmfamily]{Ligatures=TeX,Scale=1}
\fi
% Use upquote if available, for straight quotes in verbatim environments
\IfFileExists{upquote.sty}{\usepackage{upquote}}{}
\IfFileExists{microtype.sty}{% use microtype if available
  \usepackage[]{microtype}
  \UseMicrotypeSet[protrusion]{basicmath} % disable protrusion for tt fonts
}{}
\makeatletter
\@ifundefined{KOMAClassName}{% if non-KOMA class
  \IfFileExists{parskip.sty}{%
    \usepackage{parskip}
  }{% else
    \setlength{\parindent}{0pt}
    \setlength{\parskip}{6pt plus 2pt minus 1pt}}
}{% if KOMA class
  \KOMAoptions{parskip=half}}
\makeatother
\usepackage{xcolor}
\IfFileExists{xurl.sty}{\usepackage{xurl}}{} % add URL line breaks if available
\IfFileExists{bookmark.sty}{\usepackage{bookmark}}{\usepackage{hyperref}}
\hypersetup{
  pdftitle={Using GLMs to Predict Basketball Games},
  pdfauthor={Joe Despres \& Sabrina Ball},
  pdflang={en-EN},
  pdfkeywords={keyword},
  colorlinks=true,
  linkcolor=Maroon,
  filecolor=Maroon,
  citecolor=Blue,
  urlcolor=blue,
  pdfcreator={LaTeX via pandoc}}
\urlstyle{same} % disable monospaced font for URLs
\usepackage{graphicx}
\makeatletter
\def\maxwidth{\ifdim\Gin@nat@width>\linewidth\linewidth\else\Gin@nat@width\fi}
\def\maxheight{\ifdim\Gin@nat@height>\textheight\textheight\else\Gin@nat@height\fi}
\makeatother
% Scale images if necessary, so that they will not overflow the page
% margins by default, and it is still possible to overwrite the defaults
% using explicit options in \includegraphics[width, height, ...]{}
\setkeys{Gin}{width=\maxwidth,height=\maxheight,keepaspectratio}
% Set default figure placement to htbp
\makeatletter
\def\fps@figure{htbp}
\makeatother
\setlength{\emergencystretch}{3em} % prevent overfull lines
\providecommand{\tightlist}{%
  \setlength{\itemsep}{0pt}\setlength{\parskip}{0pt}}
\setcounter{secnumdepth}{-\maxdimen} % remove section numbering
% Make \paragraph and \subparagraph free-standing
\ifx\paragraph\undefined\else
  \let\oldparagraph\paragraph
  \renewcommand{\paragraph}[1]{\oldparagraph{#1}\mbox{}}
\fi
\ifx\subparagraph\undefined\else
  \let\oldsubparagraph\subparagraph
  \renewcommand{\subparagraph}[1]{\oldsubparagraph{#1}\mbox{}}
\fi
% Manuscript styling
\usepackage{upgreek}
\captionsetup{font=singlespacing,justification=justified}

% Table formatting
\usepackage{longtable}
\usepackage{lscape}
% \usepackage[counterclockwise]{rotating}   % Landscape page setup for large tables
\usepackage{multirow}		% Table styling
\usepackage{tabularx}		% Control Column width
\usepackage[flushleft]{threeparttable}	% Allows for three part tables with a specified notes section
\usepackage{threeparttablex}            % Lets threeparttable work with longtable

% Create new environments so endfloat can handle them
% \newenvironment{ltable}
%   {\begin{landscape}\begin{center}\begin{threeparttable}}
%   {\end{threeparttable}\end{center}\end{landscape}}
\newenvironment{lltable}{\begin{landscape}\begin{center}\begin{ThreePartTable}}{\end{ThreePartTable}\end{center}\end{landscape}}

% Enables adjusting longtable caption width to table width
% Solution found at http://golatex.de/longtable-mit-caption-so-breit-wie-die-tabelle-t15767.html
\makeatletter
\newcommand\LastLTentrywidth{1em}
\newlength\longtablewidth
\setlength{\longtablewidth}{1in}
\newcommand{\getlongtablewidth}{\begingroup \ifcsname LT@\roman{LT@tables}\endcsname \global\longtablewidth=0pt \renewcommand{\LT@entry}[2]{\global\advance\longtablewidth by ##2\relax\gdef\LastLTentrywidth{##2}}\@nameuse{LT@\roman{LT@tables}} \fi \endgroup}

% \setlength{\parindent}{0.5in}
% \setlength{\parskip}{0pt plus 0pt minus 0pt}

% Overwrite redefinition of paragraph and subparagraph by the default LaTeX template
% See https://github.com/crsh/papaja/issues/292
\makeatletter
\renewcommand{\paragraph}{\@startsection{paragraph}{4}{\parindent}%
  {0\baselineskip \@plus 0.2ex \@minus 0.2ex}%
  {-1em}%
  {\normalfont\normalsize\bfseries\itshape\typesectitle}}

\renewcommand{\subparagraph}[1]{\@startsection{subparagraph}{5}{1em}%
  {0\baselineskip \@plus 0.2ex \@minus 0.2ex}%
  {-\z@\relax}%
  {\normalfont\normalsize\itshape\hspace{\parindent}{#1}\textit{\addperi}}{\relax}}
\makeatother

% \usepackage{etoolbox}
\makeatletter
\patchcmd{\HyOrg@maketitle}
  {\section{\normalfont\normalsize\abstractname}}
  {\section*{\normalfont\normalsize\abstractname}}
  {}{\typeout{Failed to patch abstract.}}
\patchcmd{\HyOrg@maketitle}
  {\section{\protect\normalfont{\@title}}}
  {\section*{\protect\normalfont{\@title}}}
  {}{\typeout{Failed to patch title.}}
\makeatother
\shorttitle{Using GLMs to Predict Basketball Games}
\keywords{keyword\newline\indent Word count: X}
\usepackage{csquotes}
\ifxetex
  % Load polyglossia as late as possible: uses bidi with RTL langages (e.g. Hebrew, Arabic)
  \usepackage{polyglossia}
  \setmainlanguage[]{english}
\else
  \usepackage[main=english]{babel}
% get rid of language-specific shorthands (see #6817):
\let\LanguageShortHands\languageshorthands
\def\languageshorthands#1{}
\fi
\ifluatex
  \usepackage{selnolig}  % disable illegal ligatures
\fi
\newlength{\cslhangindent}
\setlength{\cslhangindent}{1.5em}
\newlength{\csllabelwidth}
\setlength{\csllabelwidth}{3em}
\newenvironment{CSLReferences}[2] % #1 hanging-ident, #2 entry spacing
 {% don't indent paragraphs
  \setlength{\parindent}{0pt}
  % turn on hanging indent if param 1 is 1
  \ifodd #1 \everypar{\setlength{\hangindent}{\cslhangindent}}\ignorespaces\fi
  % set entry spacing
  \ifnum #2 > 0
  \setlength{\parskip}{#2\baselineskip}
  \fi
 }%
 {}
\usepackage{calc}
\newcommand{\CSLBlock}[1]{#1\hfill\break}
\newcommand{\CSLLeftMargin}[1]{\parbox[t]{\csllabelwidth}{#1}}
\newcommand{\CSLRightInline}[1]{\parbox[t]{\linewidth - \csllabelwidth}{#1}\break}
\newcommand{\CSLIndent}[1]{\hspace{\cslhangindent}#1}

\title{Using GLMs to Predict Basketball Games}
\author{Joe Despres\textsuperscript{} \& Sabrina Ball\textsuperscript{}}
\date{}


\affiliation{\vspace{0.5cm}\textsuperscript{} Michigan State University}

\begin{document}
\maketitle

\textbf{Introduction}

Every year top university basketball teams compete in the March Madness tournament. Casual fans and enthusiasts submit predictions gambling small sums of money on the outcomes. This study will determine if which GLM performs best in predicting the results. Our data comes from three different sources Kaggle, NCAA, and the tournament results. Kagggle provides a comprehensive dataset\footnote{CITE THIS: \url{https://plexkits.com/march-madness-bracket/}} for all NCAA in season (non-turniment) basektball games from 2001 to 2020. The NCAA provides team-level statistics\footnote{CITE THIS: \url{https://stats.ncaa.org/rankings/change_sport_year_div}} for each team. We will use those two datasets to train a model that uses team-level statistics to predict basketball game outcomes. We will test our model on the actual tournament results after it is concluded on April 5th. The objective is to predict the individual game outcomes as accurately as possible.

Our response is the outcome \emph{win/loss} as a function of team statistics. Win or loss coded 0 or 1. Then we will look at both team's statistics and determine which is more important. In our dataset we have: \emph{field goal percentage} which is the rate attempt to score vs actually scoring. \emph{Free-throw percentage}, is the rate that the team has the opportunity to take an unguarded shot. Cumulative \emph{Three-point goals made}. \emph{Rebounds per game}, counts the times the team recovered the ball after a missed shot. \emph{Steals}, the times a team was able to remove possession of the ball from the opposing team. \emph{Turnover}, the number of times the team lost possession. \emph{Blocks}, The number of times the team was able to block a shot made by the opposing team.

We will use three regression methods, logistic, Poisson, and multinominal, to compare the results to determine which is most accurate. Logistic regression most naturally fits this problem because games do not tie so there is a 50/50 split of wins and losses. Also, we will try Poisson regression to predict the number of points scored by each team. Since there are many close basketball games we will use multinomial regression to predict if the game score is going to be a large difference one way, small difference, or a large difference the other way.

We Found\ldots.

\textbf{Exploratory Data Analysis}

taking a look at our variables we see that they all have a reasionably normal distribution.

\includegraphics{paper_files/figure-latex/unnamed-chunk-1-1.pdf}

\begin{longtable}[t]{lrrrrrr}
\caption{\label{tab:unnamed-chunk-2}Descriptive Statistics}\\
\toprule
Variable & Mean & Median & Std & Min & Max & Range\\
\midrule
Three-Points goals Scored & 242.08 & 241.00 & 48.18 & 107.0 & 454.00 & 347.00\\
Field Goals Scored Percentage & 44.03 & 44.10 & 2.46 & 35.4 & 52.60 & 17.20\\
Rebounds Per Game & 35.48 & 35.35 & 2.49 & 27.6 & 43.81 & 16.21\\
Steels Per Game & 206.34 & 202.00 & 40.59 & 116.0 & 369.00 & 253.00\\
Turn Overs Per Game & 421.84 & 419.00 & 49.01 & 290.0 & 604.00 & 314.00\\
\addlinespace
Blocks Per game & 3.30 & 3.20 & 0.94 & 1.3 & 6.80 & 5.50\\
\bottomrule
\end{longtable}

\textbf{Description}\\
To gather these data, we used the tidyr, (Wickham, 2021) dplyr, (Wickham, François, Henry, \& Müller, 2021) and purrr (Henry \& Wickham, 2020) package to read-in clean, and prepare our data for analysis.
Since this is a predictive model we used all the data we had at our disposal that was not colinear.

\textbf{Results}\\
\textbf{Diagnostic Methods}\\
\textbf{Conculsion}\\

\newpage

\hypertarget{references}{%
\section{References}\label{references}}

\begingroup
\setlength{\parindent}{-0.5in}
\setlength{\leftskip}{0.5in}

\hypertarget{refs}{}
\begin{CSLReferences}{1}{0}
\leavevmode\hypertarget{ref-R-purrr}{}%
Henry, L., \& Wickham, H. (2020). \emph{Purrr: Functional programming tools}. Retrieved from \url{https://CRAN.R-project.org/package=purrr}

\leavevmode\hypertarget{ref-R-tidyr}{}%
Wickham, H. (2021). \emph{Tidyr: Tidy messy data}. Retrieved from \url{https://CRAN.R-project.org/package=tidyr}

\leavevmode\hypertarget{ref-R-dplyr}{}%
Wickham, H., François, R., Henry, L., \& Müller, K. (2021). \emph{Dplyr: A grammar of data manipulation}. Retrieved from \url{https://CRAN.R-project.org/package=dplyr}

\end{CSLReferences}

\endgroup


\end{document}
